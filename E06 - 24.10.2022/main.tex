\documentclass{article}
\usepackage[utf8]{inputenc}
\usepackage{../lambdatex} %disponibile all'indirizzo http://lambdamath.altervista.it/esercizi/lambdatex.sty

\everymath{\displaystyle}
\title{Università degli Studi di Trento - Dipartimento di Matematica\\
CdL in Matematica – a.a. 2022–2023\\ Note esercitazione}
\author{Esercitatore: Simone Verzellesi\thanks{Trascrizione a cura di Davide Borra}}
\date{24 Ottobre 2022}

\begin{document}
\maketitle
\lhead{Note esercitazione}
\chead{Università degli Studi di Trento - Dipartimento di Matematica\\
CdL in Matematica – a.a. 2022–2023}
\rhead{24/10/2022}
\setlength{\headheight}{30pt}.
\begin{enumerate}[label=\textbf{Esercizio 6.\arabic*.},itemindent=*]
%%%%%%%%%%%%%%%%%%%%%%%%%%%%%%%%%%%%%%%%%%%%%%%%%%%%%
    \item Calcolare i seguenti limiti:
    \begin{enumerate}
        \item $\lim_{x\to0}\frac{\sin^4 x}{(1-\cos x)^3}$
        \item $\lim_{n\to+\infty}\left( \frac{1+n}{3+n} \right)^{n^2}$
        \item $\lim_{x\to+\infty}\left( \frac{1}{3x} \right)^{\frac{1}{\log(2x)}}$
        \item $\lim_{x\to0^+}\left( \frac{2^x+3^x}{2} \right)^\frac{1}{x}$
    \end{enumerate}
    \item[\textit{\large Soluzione~}]~
    \begin{enumerate}
        \item $\lim_{x\to0}\frac{\sin^4 x}{(1-\cos x)^3}=\left[ \frac{0}{0} \right]=\lim_{x\to 0}\underbrace{\frac{\sin x^4}{x^4}}_{\to 1}\cdot \underbrace{\frac{x^6}{(1-\cos x)^3}}_{\to \frac{1}{8}}\cdot \underbrace{\frac{x^4}{x^6}}_{\to +\infty}=+\infty$
        \item $\lim_{n\to+\infty}\left( \frac{1+n}{3+n} \right)^{n^2}=\left[ 1^\infty\right]=0^+$
        \[\left(\frac{1+n}{3+n}\right)^{n^2}=\left(\frac{\cancel{n}\left( 1+\frac{1}{n} \right)^n}{\cancel{n}\left( 1+\frac{3}{n} \right)^n}\right)^{n}\to \left( \frac{1}{e^2} \right)^n\to 0^+\]
        \item $\lim_{x\to+\infty}\left( \frac{1}{3x} \right)^{\frac{1}{\log(2x)}}=\left[ {0}^{0} \right]=\lim_{x\to+\infty}e^{\log\left( \frac{1}{3x} \right)^{\frac{1}{\log(2x)}}}=\lim_{x\to+\infty}e^{\frac{-\log(3x)}{\log(2x)}}=\lim_{x\to+\infty}e^{-\frac{\log 3+\log x}{\log 2+\log x}}=e^{-1}$
        \item $\lim_{x\to0^+}\left( \frac{2^x+3^x}{2} \right)^\frac{1}{x}=\left[ 1^\infty\right]=e^{\frac{\log2}{2}+\frac{\log3}{2}}=\sqrt{6}$
        \[\left( \frac{2^x+3^x}{2} \right)^\frac{1}{x}=e^{\log\left( \frac{2^x+3^x}{2} \right)^\frac{1}{x}}=e^{\frac{1}{x}\log\left( \frac{2^x+3^x}{2} \right)}=e^{\overbrace{\frac{\log\left( 1+\frac{2^x+3^x}{2}-1 \right)}{\frac{2^x+3^x}{2}-1}}^{\to1}\cdot\overbrace{\frac{\frac{2^x+3^x}{2}-1}{x}}^{\to\frac{\log2}{2}+\frac{\log3}{2}}}\to e^{\frac{\log2}{2}+\frac{\log3}{2}}\]
        \[\frac{\frac{2^x+3^x}{2}-1}{x}=\underbrace{\frac{2^x-1}{2x}}_{\to\frac{\log2}{2}}+\underbrace{\frac{3^x-1}{2x}}_{\to\frac{\log3}{2}}\to \frac{\log2}{2}+\frac{\log3}{2}\]
    \end{enumerate}
    %%%%%%%%%%%%%%%%%%%%%%%%%%%%%%%%%%%%%%%%%%%%%%%%%
    \item Calcolare al variare di $\lambda\in \R$
    \begin{enumerate}
        \item $\lim_{x\to 0^+}\frac{\tg^22x-\sin^22x}{x^\gamma}$
        \item $\lim_{x\to +\infty}\left( \frac{3+x^\gamma}{2+x} \right)^x$
    \end{enumerate}
    \item[\textit{\large Soluzione~}]~
    \begin{enumerate}
        \item $\lim_{x\to 0^+}\frac{\tg^22x-\sin^22x}{x^\gamma}$
        \begin{itemize}
            \item Se $\gamma=0$, si ha $\lim_{x\to 0^+}{\tg^22x-\sin^22x}=0$
            \item Se $\gamma<0$, si ha $\lim_{x\to 0^+}{x^{-\gamma}(\tg^22x-\sin^22x)}=0$
            \item Se $\gamma>0$, si ha una forma di indecisione $\left[ \frac{0}{0} \right]$
            \[
                \frac{\tg^22x-\sin^22x}{x^\gamma}=\frac{\sin^22x\left( \frac{1}{\cos^2x}-1 \right)}{x^\gamma}=\frac{\sen^22x}{x^\gamma}\underbrace{\left( \frac{1-\cos^22x}{\cos^22x} \right)}_{\tg^2 x}=\frac{\sin^22x\tg^22x}{x^\gamma}=
            \]
            \[
                =\underbrace{\left( \frac{\sin2x}{2x} \right)}_1\underbrace{\left( \frac{\sin2x}{2x}\right)}_1\cdot 16x^{4-\gamma}
            \]
            Quindi si ha
            \[\lim_{x\to 0^+}\frac{\tg^22x-\sin^22x}{x^\gamma}=\begin{cases}
                0&\text{se }\gamma<4\\
                16&\text{se }\gamma=4\\
                +\infty&\text{se }\gamma >4\\
            \end{cases}\]
        \end{itemize}
        \item $\lim_{x\to +\infty}\left( \frac{3+x^\gamma}{2+x} \right)^x$
        \begin{itemize}
            \item Se $\gamma<1$ domina il denominatore, quindi $\lim_{x\to +\infty}\left( \frac{3+x^\gamma}{2+x} \right)^x=0$
            \item Se $\gamma>1$ domina il numeratore, quindi $\lim_{x\to +\infty}\left( \frac{3+x^\gamma}{2+x} \right)^x=+\infty$
            \item Se $\gamma=1$, si ha una forma di indecisione 
            \[\left( \frac{3+x}{2+x} \right)^x=\left( \frac{1+\frac{3}{x}}{1+\frac{2}{x}} \right)^x=\frac{\left( {1+\frac{3}{x}}\right)^x}{\left({1+\frac{2}{x}} \right)^x}\to\frac{e^3}{e^2}=e\]
        Quindi si ha
            \[\lim_{x\to +\infty}\left( \frac{3+x^\gamma}{2+x} \right)^x=\begin{cases}
                0&\text{se }\gamma<1\\
                e&\text{se }\gamma=1\\
                +\infty&\text{se }\gamma >1\\
            \end{cases}\]
        \end{itemize}
        
    \end{enumerate}
%%%%%%%%%%%%%%%%%%%%%%%%%%%%%%%%%%%%%%%%%%%%%%%%%%%%%
    \item Sia $(a_n)_n$ una successione definita per ricorrenza come 
    \[\begin{cases}
        a_1=1\\
        a_n=\sqrt{1+a_n}
    \end{cases}\]
    Dimostrare che $\exists \lim_{n\to+ \infty}a_n$ finito e calcolarlo.
    \item[\textit{\large Soluzione~}]~
    Per dimostrare che esiste limite basta dimostrare che la successione è monotona.
    Prima di tutto facciamoci un'idea di quale possa essere la monotonia, magari calcolando i primi termini
    \[a_1=1~~~~a_2=\sqrt{2}~~~\Rarr~~~a_1<a_2\]
    Potrebbe essere crescente? Dimostriamolo per induzione:
    \begin{itemize} 
        \item Passo base: $a_1\leq a_2~~\Larr~~a_1=1~~~a_2=\sqrt{2}$
        \item Passo induttivo:  $a_{n+1}\geq a_n ~~\Rarr~~ a_{n+2}\geq a_{n+1}$
        \[\Harr ~~ \sqrt{1+a_{n+1}}\geq \sqrt{1+a_n}~~\Harr~~\cancel{1}+a_{n+1}\geq\cancel{1}+a_n~~~\Harr~~~a_{n+1}\geq a_n\]
    \end{itemize}
    La funzione è effettivamente crescente, quindi per il teorema dell'esistenza del limite per successioni monotone, $\exists \lim_{n\to+\infty}a_n$. Per dimostrare che il limite è finito, bisogna dimostrare che $(a_n)_n$ è limitata. Siccome è crescente, è limitata inferiormente. Dimostriamo che è limitata superiormente, ovvero che $\exists M:a_n\leq M~\forall n$. Dato un $M\geq 1$ arbitrario, provare che \[\forall n, ~a_n\leq M\]. Usiamo il principio di induzione:
    \begin{itemize} 
        \item Passo base: $(n=1) ~~a_1\leq M~~\Larr~~a_1=1\leq M$
        \item Passo induttivo:  $a_n\leq M ~~\Rarr~~ a_{n+1}\leq M$
        \[\Harr ~~ \sqrt{1+a_{n}}\leq M~~\Harr~~1+a_{n}\leq M
        ^2~~~\Harr~~~a_{n}\leq M
        ^2-1\]
        Ci basta quindi dimostrare che $a_n\leq M ~~\Rarr~~ a_{n}\leq M^2-1$. Basta scegliere un qualsiasi $M$ tale che $M\leq M^2-1$, ad esempio $M=2$. 
    \end{itemize}
    Di conseguenza la successione è limitata superiormente e $2$ è un suo maggiorante, per cui $\exists \lim_{n\to+ \infty}a_n$ finito.
    \begin{oss}
        L'insieme degli $M$ che abbiamo appena individuato è un insieme dei maggioranti di $(a_n)_n$, ma non è detto che siano \emph{tutti} i maggioranti.
    \end{oss}
    Calcoliamo quindi il limite. Se esiste (devo averlo dimostrato precedentemente!) \[l=\lim_{n\to+\infty}a_n=\lim_{n\to+\infty}a_{n+1}\]
    \[a_{n+1}=\sqrt{1+a_n}~~\overset{n\to+\infty}\Longrightarrow~~l=\sqrt{1+l}~~\overset{l\geq0}\Longrightarrow~~l^2=1+l~~\Harr\]
    \[\Harr~~l=\frac{1+\sqrt{5}}{2}=\langle\begin{array}{l}
        \boxed{1,6180339887\dots=\varphi}\\
        \cancel{-0,6180339887\dots=\Phi}
    \end{array}\]
    \[l=\varphi\]
%%%%%%%%%%%%%%%%%%%%%%%%%%%%%%%%%%%%%%%%%%%%%%%%%%%%%
    \item Sia $f: \R\to \R$ definita come \[f(x)=\begin{cases}
        \frac{\sin x^2}{x(\sqrt{x+1}-1)}&\text{se }x>0\\
        ke^x-2 & \text{se }x\leq 0
    \end{cases}\]
    Determinare $k$ affinché $f$ sia continua su tutto $\R$.
    \item[\textit{\large Soluzione~}]~
    La funzione è continua in $\R\setminus 0$ in quanto i tratti sono ottenuti come somma, prodotto e composizione di funzioni continue. Studiamo quindi la continuità in $x=0$.
    \[\lim_{x\to0^-}f(x)=\lim_{x\to0^-}(ke^x-2)=k-2=f(0)\]
    \[\lim_{x\to0^+}\frac{\sin x^2}{x(\sqrt{x+1}-1)}=2\]
    \[\frac{\sin x^2}{x(\sqrt{x+1}-1)}=\frac{\sin x^2}{x^2}\cdot x\cdot \frac{1}{x(\sqrt{x+1}-1)}\to\frac{\cancel{x}(\sqrt{x+1}+1)}{\cancel{x}+\bcancel{1}-\bcancel{1}}=\sqrt{x+1}+1\to 2\]
    Impongo che $\lim_{x\to0^-}f(x)=f(0)=\lim_{x\to0^+}f(x)$:
    \[k-2=2~~~\Rarr~~~k=4\]
%%%%%%%%%%%%%%%%%%%%%%%%%%%%%%%%%%%%%%%%%%%%%%%%%%%%%
\item ~
\begin{enumerate}
    \item Sia $f(x)=x^5+2x^3-2$. Dimostrare che $\exists!x_0\in \R:f(x_0)=0$ e determinare un intervallo $]a,b[$ con $\scriptstyle b-a\leq\frac{1}{4}$ tale che $x_0\in]a,b[$
    \item Sia $f(x)=x^3+x+1$ su $\R$. Dimostrare che $f$ è invertibile.
    \item Sia $f:\R\to\R$. Dimostrare che se $\exists ]a,b[\subseteq \R$
\end{enumerate}
\item[\textit{\large Soluzione~}]~
%%%%%%%%%%%%%%%%%%%%%%%%%%%%%%%%%%%%%%%%%%%%%%%%%%%%%

\end{enumerate}
\end{document}
