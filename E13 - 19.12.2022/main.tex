\documentclass{article}
\usepackage{../../LaTeX/lambdatex} %disponibile all'indirizzo http://lambdamath.altervista.it/esercizi/lambdatex.sty
\usepackage{tasks}
\usepackage{exsheets}
\newcommand{\se}{\text{ se }}
\renewcommand{\phi}{\varphi}
\everymath{\displaystyle}


\title{Università degli Studi di Trento - Dipartimento di Matematica\\
CdL in Matematica – a.a. 2022–2023\\ Note esercitazione}
\author{Esercitatore: Simone Verzellesi\thanks{Trascrizione a cura di Davide Borra}}
\date{19 Dicembre 2022}
\begin{document}
\maketitle
\lhead{Note esercitazione}
\chead{Università degli Studi di Trento - Dipartimento di Matematica\\
CdL in Matematica – a.a. 2022–2023}
\rhead{19/12/2022}
\setlength{\headheight}{30pt}
\subsection*{Equazioni differenziali a variabili separabili}
\begin{enumerate}[label=\textbf{Esercizio 13.\arabic*.},itemindent=*]
%%%%%%%%%%%%%%%%%%%%%%%%%%%%%%%%%%%%%%%%%%%%%%%%%%%%%
\item Risolvere il seguente problema di Cauchy
\[\begin{cases}
    y'=-\frac{x}{y}\\
    y(0)=1
\end{cases}.\]

\item[\textit{\large Soluzione~}]~
Osserviamo che ogni soluzione $y$ sarà $y(x)\neq 0 ~\forall x$. Allora
\[y'=-\frac{x}{y}~ ~ \Harr ~~ y'y=- x\]
Abbiamo un'equazione differenziale a variabili separabili. Integrando si ha 
\[\int y'y\d x=-\int x \d x~~~\Harr~~~\int y\d y =-\int x \d x~~\Harr~~~\frac{1}{2}y^2=-\frac{1}{2}x^2+c~~~\Harr\]
\[\Harr~~~y(x)=\sqrt{-x^2+c}~~\lor~~ y=-\sqrt{-x^2+c}~~~\text{con}~-\sqrt{c}\leq x\leq \sqrt{c},~ ~ c\geq 0\]
Per risolvere il problema di Cauchy impongo $y(0)=1$, quindi
\[\pm\sqrt{c}=1\implies c=1\]
Scegliendo la soluzione con segno positivo (deve soddisfare $y(0)=1>0$), per cui la soluzione al problema di Cauchy dato è la funzione $y:]-1,1[\to \R~~~y(x)=\sqrt{1-x^2}$
%%%%%%%%%%%%%%%%%%%%%%%%%%%%%%%%%%%%%%%%%%%%%%%%%%%%%
\item Determinare l'integrale generale di 
\[
    (x^2+1)y'+y^2=0   
\]
e risolvere il problema di Cauchy
\[\begin{cases}
    (x^2+1)y'+y^2=0 \\
    y(0)=1
\end{cases}\]
\item[\textit{\large Soluzione~}]~ Dobbiamo riportarci alla forma $\frac{y'}{g(y)}=h(x)$, ma per farlo dobbiamo poter dividere per $y$. Cerchamo quindi le soluzioni banali: Supponiamo $y$ costante, allora $y'=0$, di conseguenza
\[y^2\equiv 0~~\Harr~~y\equiv 0\]
Abbiamo che $y=0$ è soluzione dell'equazione differenziale data, ma non soluzione del problema di Cauchy. Tutte le altre soluzioni non si annullano mai per il teorema di Cauchy.\\
Possiamo quindi procedere
\[\frac{y'}{y}=-\frac{1}{1+x^2}~~~\Rarr~~~\int\frac{y'}{y^2}\d x=-\int\frac{1}{1+x^2}\d x~~~\Harr~~~\int\frac{1}{y^2}\d y=-\arctg x+c~~~\Harr\]
\[-\frac{1}{y}=-\arctg x +c~~~\Harr~~~y(x)=\frac{1}{\arctg x +c}~~~~y: \R\setminus \{\tg c\}\to \R\]

Per risolvere il PdC imponiamo la condizione iniziale.
\[1=\frac{1}{\arctg 0 +c}~~~\Harr~~~c=1~~\implies~~y(x)=\frac{1}{\arctg x +1} \text{ con }x\in \R\setminus \{\tg (-1)\}\]
Siccome la soluzione al problema di Cauchy è definita su un intervallo, che deve contenere lo $0$ e $\tg (-1)<0$, scelgo l'intervallo $x>\tg(-1)$.
%%%%%%%%%%%%%%%%%%%%%%%%%%%%%%%%%%%%%%%%%%%%%%%%%%%%%
\item Trovare l'integrale generale di 
\[
    y'=xy^2
\]
e risolvere i problemi di Cauchy con
\begin{tasks}(2)
    \task $y(0)=1$ \task $y(0)=-1$.
\end{tasks}
\item[\textit{\large Soluzione~}]~ Troviamo la soluzione banale $y=0$, quindi è lecito assumere per le altre soluzioni $y\neq 0$
\[\int \frac{y'}{y^2}\d x=\int x \d x~~~\Harr~~~-\frac{1}{y}=\frac{x^2}{2}+c~~~\Harr~~~y=-\frac{2}{x^2+2c}\]
Per determinare il dominio dobbiamo separare i tre casi
\[\dom y=\begin{cases}
    \R &\se c>0\\
    \R\setminus \{0\}&\se c=0\\
    \R\setminus \{-\sqrt{-2c}, \sqrt{-2c}\}&\se c< 0.
\end{cases}\]
Infine
\begin{enumerate}
    \item Imponiamo la condizione $y(0)=1$. Allora $1=-\frac{2}{2c}~~\Harr~~c=-1$. La soluzione e quindi definita sull'intervallo $]-\sqrt{-2c}, \sqrt{-2c}[$
    \item Imponiamo la condizione $y(0)=-1$ e si ha $c=1$. Quindi la soluzione $y$ è definita su $\R$.
\end{enumerate}
%%%%%%%%%%%%%%%%%%%%%%%%%%%%%%%%%%%%%%%%%%%%%%%%%%%%%
\end{enumerate}
\vspace{1cm}
\subsection*{Equazioni differenziali lineari del primo ordine}

\begin{lineframe}
    \paragraph*{Ricorda} In generale, data l'equazione differenziale $y'=a(x)y+b(x)$ con $a(x),b(x)\in \mathcal{C}^0(I)$, $I$ intervallo, il suo integrale generale è una funzione del tipo 
    \[y(x)=e^{A(x)}\left( c+\int b(x)e^{-A(x)}\d x \right),\]
    dove $A(x)=\int a(x)\d x$
\end{lineframe}

\begin{enumerate}[label=\textbf{Esercizio 13.\arabic*.},itemindent=*]
\setcounter{enumi}{3}
%%%%%%%%%%%%%%%%%%%%%%%%%%%%%%%%%%%%%%%%%%%%%%%%%%%%%
\item Risolvere il seguente problema di Cauchy
\[\begin{cases}
    y'-y=1\\
    y(0)=0
\end{cases}\]
\item[\textit{\large Soluzione~}]~Riscriviamo l'equazione differenziale nella forma $y'=a(x)y+b(x)$, dove $a(x)=1$ e $b(x)=1$. Poichè
$A(x)=\int1\d x=x$, la formula sopra ci dà
\[y(x)=e^x\left( c+\int e^{-x}\d x \right)=e^{x}\left( c +(-e^{-x}) \right)=ce^x-1.\]
Impongo la condizione $y(0)=0$. Allora
\[y(0)=c-1 = 0~~~\Harr~~~c=1~~~\implies ~~~y(x)=e^{x}-1~~~~~x\in \R\]
è la soluzione del problema di Cauchy dato.
%%%%%%%%%%%%%%%%%%%%%%%%%%%%%%%%%%%%%%%%%%%%%%%%%%%%%
\item Risolvere il seguente problema di Cauchy
\[\begin{cases}
    y'=\left( \sin x \right)y+\sin(2x)\\
    y(0)=-2
\end{cases}\]
\item[\textit{\large Soluzione~}]~ Abbiamo un'equazione differenziale lineare del 1° ordine con $a(x)=\sin x$ e $b(x)=\sin (2x)$. Ora
\[a(x)=\sen x~~~\implies~~~A(x)=\int a(x)\d x=\int \sen x \d x=-\cos x\]
L'integrale generale si ricava quindi da 
\[y(x)=e^{ -\cos x}\left( c+\int \sin (2x) e^{\cos x} \d x \right).\]
Ora
\[\int \sen(2x)e^{\cos x}\d x=\int 2\cos x \sin x e^{\cos x}\d x.\]
Pongo $t=\cos x$, quindi $\d t=-\sin x \d x$;
\[\int_2\cos x \sin xe^{\cos x}\d x \underset{t =\cos x}{= } -2\int te^t\d t=-2\left( te^t-\int e^t\d t \right)=-2\left(te^t-e^t\right)=-2e^{\cos x}\cos x+2e^{\cos x}\]
Allora l'integrale generale dell'equazione è
\[y(x)=e^{-\cos x}\left( c+2e^{\cos x}-2\cos xe^{\cos x} \right)=ce^{-\cos x}+2-2\cos x, c\in \R\]
Imponiamo le condizioni $y(0)=-2$ e otteniamo
\[ce^{-1}=-2~~~\Harr~~~c=-2e .\]
%%%%%%%%%%%%%%%%%%%%%%%%%%%%%%%%%%%%%%%%%%%%%%%%%%%%%
\item Risolvere il seguente problema di Cauchy
\[\begin{cases}
    y'=y^2x\log x^2\\
    y(1)=\lambda&\lambda \in \R
\end{cases}\]
\item[\textit{\large Soluzione~}]~
Cerchiamo la soluzione banale. Se $y'\equiv 0$, allora $y\equiv 0$. Essa è soluzione del problema di Cauchy se $\lambda=0$.
Per trovare le altre soluzioni è quindi lecito assumere $y\not\equiv 0$ e $\lambda \neq 0$.
Abbiamo

\[\frac{y'}{y^2}=x\log x^2\]
\[\int \frac{y'}{y^2}\d x=\int x\log x^2 \d x~\underset{\text{pongo } t=x^2~\Rarr~\d t=2x\d x}{\Harr}~-\frac{1}{y}=\frac{1}{2}\int \log t \d t .\]
Ora
\[\int \log t\d t =t \log t-\int \frac{t}{t}\d t=t(\log t-1);\]
ricordando che $t=x^2$ si ottiene infine
\[-\frac{1}{y}={\frac{1}{2}x^2\log x^2-\frac{1}{2}x^2+c},\quad c\in \R\]
\[y=-\frac{1}{\frac{1}{2}x^2\log x^2-\frac{1}{2}x^2+c}.\]
Per risolvere il problema di Cauchy imponiamo $y(1)=\lambda$:
\[\lambda =-\frac{1}{c-\frac{1}{2}}~~~\Harr~~~\frac{1}{2}-c=\frac{1}{\lambda}~~~\Harr~~~c=\frac{1}{2}-\frac{1}{\lambda}.\]
%%%%%%%%%%%%%%%%%%%%%%%%%%%%%%%%%%%%%%%%%%%%%%%%%%%%%
\end{enumerate}
\end{document}
