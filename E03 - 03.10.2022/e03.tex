\documentclass{article}
\usepackage[utf8]{inputenc}
\usepackage{../lambdatex} %disponibile all'indirizzo http://lambdamath.altervista.it/esercizi/lambdatex.sty

\everymath{\displaystyle}
\title{Università degli Studi di Trento - Dipartimento di Matematica\\
CdL in Matematica – a.a. 2022–2023\\ Note esercitazione}
\author{Esercitatore: Simone Verzellesi\thanks{Trascrizione a cura di Davide Borra e Matilde Calabri}}
\date{03 Ottobre 2022}

\begin{document}
\maketitle
\lhead{Note esercitazione}
\chead{Università degli Studi di Trento - Dipartimento di Matematica\\
CdL in Matematica – a.a. 2022–2023}
\rhead{03/10/2022}
\begin{enumerate}[label=\textbf{Esercizio 3.\arabic*.},itemindent=*]
%%%%%%%%%%%%%%%%%%%%%%%%%%%%%%%%%%%%%%%%%%%%%%%%%%%%%
    \item Determinare gli estremi inferiore e superiore ed eventuali massimi e minimi dei seguenti insiemi 
    \begin{enumerate}
        \item $A=\left\{\frac{n}{n+1},n\in \N\right\}$
        \item $B=\left\{2(-1)^n-\frac{1}{2n}, n\in \N\setminus \{0\}\right\}$
        \item $C=\left\{\frac{n}{m}: n, m\in \N\setminus\{0\},n>m\right\}$
    \end{enumerate}
    \item[\textit{\large Soluzione~}]~
    \begin{enumerate}
        \item \begin{oss}
            Possiamo riscrivere $\frac{n}{n+1}=\frac{n+1-1}{n+1}=1-\frac{1}{n+1}$.
        \end{oss}
        Se calcoliamo alciuni elementi dell'insieme $A=\left\{0, \frac{1}{2}, \frac{2}{3}, \frac{3}{4},\dots\right\}$ ci accorgiamo che sono tutti compresi in $[0,1[$.\\
        \textbf{Estremo inferiore:} siccome $n\in \N$, $\frac{n}{n+1}>=0\forall n\in \N$. Inoltre $0\in A$, per cui si ha che $\inf A=0$ e $\min A=0$.\\
        \textbf{Estremo superiore:} Dobbiamo dimostrare che $\sup A=1$
        \begin{itemize}
            \item $1\geq 1-\frac{1}{n+1}$ perché $\frac{1}{n+1}>0~~\forall n\in \N$
            \item Dobbiamo provare che $\forall \varepsilon >0, \exists n\in \N:\frac{n}{n+1}\geq 1-\varepsilon$. Fissiamo $\varepsilon$ e cerchiamo $n$: \[\frac{n}{n+1}\geq 1-\varepsilon~~~\Harr\]\[\cancel{1}-\frac{1}{n+1}=\cancel{1}-\varepsilon~~~\Harr\]\[(n+1)\varepsilon\geq 1~~~\Harr\]\[n\varepsilon\geq 1-\varepsilon ~~~\Harr\]\[n\geq \frac{1-\varepsilon}{\varepsilon}\]
            Se $\varepsilon \geq 1$, la disequazione è sempre verificata perché il numeratore diventa negativo. \\Se $varepsilon<1$ basta scegliere un numero naturale $n>\frac{1-\varepsilon}{\varepsilon}$.\\Questo dimostra che $\sup A=1$. L'insieme non ammette massimo perché $1\notin A$.
        \end{itemize}
        \item Separiamo i casi in cui $n$ è pari da quelli in cui $n$ è dispari:
        \[B=\underbrace{\left\{2-\frac{1}{2n}, \in \N \text{ e }n\text{ pari}\right\}}_{B'}\cup\underbrace{\left\{-2-\frac{1}{2n}, \in \N \text{ e }n\text{ dispari}\right\}}_{B''}\]
        \begin{oss}
            È un'unione disgiunta, cioè $\forall x \in B', y\in B'', y<0<x$, infatti 
                \[2-\frac{1}{2n}>0\text{~~ e ~~}-2-\frac{1}{2n}<0\]
            Da questo si ottiene che $\sup B=\sup B'$ e $\inf B=\inf B''$     
        \end{oss}
        \textbf{Estremo superiore:}
        \begin{oss}
            In $B'$ gli elementi si avvicinano a 2 quando $n$ cresce perché $\frac{1}{2n}$ diventa sempre più piccolo. 
        \end{oss}
        La dimostrazione è analoga a quella del $\sup A$.\\
        \textbf{Estremo inferiore:}\\
        Calcoliamo alcuni degli elementi di $B=\left\{-\frac{5}{2},-\frac{13}{6},\dots\right\}$. Si nota che all'aumentare di $n$ gli elementi di $B''$ crescono, di conseguenza si ha $\inf B=\min B''= -\frac{5}{2}$.
        \item \textbf{Estremo superiore:}
        \begin{oss}
            Se pongo $m=1$, ottengo $\N_{\geq2}$. Di conseguenza $\N_{\geq2}\subseteq C$. Da questo si ricava che $\sup C\geq \sup \N_{\geq 2}$, da cui (siccome $\sup \N_{\geq 2}=+\infty$), \[\sup C=+\infty\]
        \end{oss}
        \textbf{Estremo inferiore:}\\
        Scelgo $m=n+1$, ottengo $D=\left\{\frac{n+1}{n}, n\neq 0\right\}\subseteq C$. Analogamente a quanto fatto nel punto (a) è possibile dimostrare che $\inf D=1$. Inoltre siccome $D\subseteq C$, si ha $\inf D\geq \inf C\Rarr \inf C\leq 1$. Inoltre sappiamo che $\forall n> m, \frac{n}{m}>$, di conseguenza $\inf C\geq 1$. Siccome $\inf C\geq 1$ e $\inf C \leq 1$, si ha $\inf C=1$.
    \end{enumerate}
    
%%%%%%%%%%%%%%%%%%%%%%%%%%%%%%%%%%%%%%%%%%%%%%%%%%%%%
    \item Risolvere in $\C$ le seguenti equazioni: 
        \begin{enumerate}
            \item $\frac{1}{|z|\cdot \overline{z}}=\cos\frac{\pi}{4}+i\sen \frac{\pi}{4}$
            \item $\overline{z}(\Im z-\Re z)=z$
            \item $z^3=1$
            \item $2z^2+\overline{z}=-1$
        \end{enumerate}
    \item[\textit{\large Soluzione~}]
    Lorem ipsum dolor sit amet, consectetur adipiscing elit. Mauris vel augue imperdiet, molestie mauris non, euismod nunc. Cras finibus sit amet ipsum sit amet placerat. Etiam in pulvinar risus. Aliquam sed feugiat magna, ac dapibus velit. Pellentesque vestibulum fermentum leo, sit amet hendrerit nisl tempor sed. Cras quis enim et urna faucibus venenatis at ac nulla. Mauris ac egestas lorem. Aenean mollis, libero ac sollicitudin faucibus, erat nunc ultricies odio, nec vehicula nisl purus tincidunt tellus. Maecenas interdum blandit dui, in luctus ipsum posuere vitae. Pellentesque ac accumsan velit, non accumsan odio. Vivamus aliquet cursus velit at rhoncus. Donec laoreet pellentesque accumsan. Aliquam pulvinar ac massa sit amet pulvinar. Duis tristique, magna ut blandit congue, erat ligula dignissim turpis, vel pharetra lectus massa faucibus nisl.Lorem ipsum dolor sit amet, consectetur adipiscing elit. Mauris vel augue imperdiet, molestie mauris non, euismod nunc. Cras finibus sit amet ipsum sit amet placerat. Etiam in pulvinar risus. Aliquam sed feugiat magna, ac dapibus velit. Pellentesque vestibulum fermentum leo, sit amet hendrerit nisl tempor sed. Cras quis enim et urna faucibus venenatis at ac nulla. Mauris ac egestas lorem. Aenean mollis, libero ac sollicitudin faucibus, erat nunc ultricies odio, nec vehicula nisl purus tincidunt tellus. Maecenas interdum blandit dui, in luctus ipsum posuere vitae. Pellentesque ac accumsan velit, non accumsan odio. Vivamus aliquet cursus velit at rhoncus. Donec laoreet pellentesque accumsan. Aliquam pulvinar ac massa sit amet pulvinar. Duis tristique, magna ut blandit congue, erat ligula dignissim turpis, vel pharetra lectus massa faucibus nisl.

%%%%%%%%%%%%%%%%%%%%%%%%%%%%%%%%%%%%%%%%%%%%%%%%%%%%%
    \item Si considerino le funzioni $f, g, h: \C\to\C$ definite come \[f(z)=z+z_0 \text{, ~~ con $z_0\in \C$ fissato}\]
    \[g(z)=iz\]
    \[h(z)=z^2\]
    e gli insiemi \[A=\{z: 0\leq \Re z \leq 1 \text{ e } 0\leq \Im z \leq 1\}\]
    \[B=\left\{z:|z|\leq 2, \arg z\in \left]\frac{3}{8}\pi, \frac{\pi}{2}\right[\right\}\]
    \[C=\{z:1\leq |z|\leq 2\}\]
    Rappresentare $f(A), g(B), h(B), h(C)$.
    \item[\textit{\large Soluzione~}] Prima di tutto rappresentiamo gli insiemi $A$, $B$, $C$.
    \begin{figure}[h!]
        \centering
        \begin{subfigure}{0.3\textwidth}
            \definecolor{zzttqq}{rgb}{0.6,0.2,0.}
            \begin{tikzpicture}[line cap=round,line join=round,>=triangle 45,x=2cm,y=2cm]
                \begin{axis}[
                x=2.0cm,y=2.0cm,
                axis lines=middle,
                ymajorgrids=true,
                xmajorgrids=true,
                xmin=-0.6,
                xmax=1.6,
                ymin=-0.6,
                ymax=1.6,
                xtick={-0.5,0.0,...,1.5},
                ytick={-0.5,0.0,...,1.5},]
                \clip(-0.6,-0.6) rectangle (1.6,1.6);
                \fill[line width=2.pt,color=zzttqq,fill=zzttqq,fill opacity=0.10000000149011612] (0.,1.) -- (0.,0.) -- (1.,0.) -- (1.,1.) -- cycle;
                \draw [line width=2.pt,color=zzttqq] (0.,1.)-- (0.,0.);
                \draw [line width=2.pt,color=zzttqq] (0.,0.)-- (1.,0.);
                \draw [line width=2.pt,color=zzttqq] (1.,0.)-- (1.,1.);
                \draw [line width=2.pt,color=zzttqq] (1.,1.)-- (0.,1.);
                \draw (0.29307222648774567,0.9178337611061765) node[anchor=north west] {$A$};
                \end{axis}
                \end{tikzpicture}
        \end{subfigure}
        \begin{subfigure}{0.3\textwidth}
            \definecolor{zzttqq}{rgb}{0.6,0.2,0.}
            \definecolor{qqwuqq}{rgb}{0.,0.39215686274509803,0.}
            \begin{tikzpicture}[line cap=round,line join=round,>=triangle 45,x=1.0cm,y=1.0cm, scale = 1.1]
            \begin{axis}[
            x=1.0cm,y=1.0cm,
            axis lines=middle,
            ymajorgrids=true,
            xmajorgrids=true,
            xmin=-1.5,
            xmax=2.5,
            ymin=-1.5,
            ymax=2.5,
            xtick={-1.0,0.0,...,2.0},
            ytick={-1.0,0.0,...,2.0},]
            \clip(-1.5,-1.5) rectangle (2.5,2.5);
            \draw [line width=2.pt,color=qqwuqq,fill=qqwuqq,fill opacity=0.10000000149011612] (0,0) -- (0.:0.3741774256653813) arc (0.:67.5:0.3741774256653813) -- cycle;
            \draw [line width=2.pt,color=zzttqq,fill=zzttqq,fill opacity=0.10000000149011612]  (0,0) --  plot[domain=1.1780972450961724:1.5707963267948966,variable=\t]({1.*2.*cos(\t r)+0.*2.*sin(\t r)},{0.*2.*cos(\t r)+1.*2.*sin(\t r)}) -- cycle ;
            \draw [color=qqwuqq](0.4888435380798806,0.8070387323953816) node[anchor=north west] {$\vartheta =\frac{3}{8}\pi$};
            \draw (0.05,1.5553935837261437) node[anchor=north west] {$B$};
            \end{axis}
            \end{tikzpicture}
        \end{subfigure}
        \begin{subfigure}{0.3\textwidth}
            \definecolor{qqqqff}{rgb}{0.,0.,1.}
            \begin{tikzpicture}[line cap=round,line join=round,>=triangle 45,x=1.0cm,y=1.0cm]
            \begin{axis}[
            x=1.0cm,y=1.0cm,
            axis lines=middle,
            ymajorgrids=true,
            xmajorgrids=true,
            xmin=-2.5,
            xmax=2.5,
            ymin=-2.5,
            ymax=2.5,
            xtick={-2.0,-1.0,...,2.0},
            ytick={-2.0,-1.0,...,2.0},]
            \clip(-2.5,-2.5) rectangle (2.5,2.5);
            \filldraw [even odd rule, line width=2.pt,color=qqqqff,fill=qqqqff,fill opacity=0.10000000149011612] (0,0) circle (2) (0,0) circle (1);
            \node[color=qqqqff] at (1,1) {$C$};
            \end{axis}
            \end{tikzpicture}
        \end{subfigure}
    \end{figure}
    Per come funziona la somma in $C$ (metodo punta-coda), sommare $z_0$ ad ogni punto dell'intervallo $A$ è come applicare alla rappresentazione dell'insieme una traslazione di vettore $v$.\\
    \begin{figure}[h!]
        \centering
        \definecolor{zzttqq}{rgb}{0.6,0.2,0.}
        \begin{tikzpicture}[line cap=round,line join=round,>=triangle 45,x=2.0cm,y=2.0cm]
        \begin{axis}[
        x=2.0cm,y=2.0cm,
        axis lines=middle,
        ymajorgrids=true,
        xmajorgrids=true,
        xmin=-0.37507864260047524,
        xmax=3.756772840196521,
        ymin=-0.40873055254144736,
        ymax=2.2923711155603104,
        xtick={-0.0,0.5,...,3.5},
        ytick={-0.0,0.5,...,2.0},]
        \clip(-0.37507864260047524,-0.40873055254144736) rectangle (3.756772840196521,2.2923711155603104);
        \fill[line width=2.pt,color=zzttqq,fill=zzttqq,fill opacity=0.10000000149011612] (0.,1.) -- (0.,0.) -- (1.,0.) -- (1.,1.) -- cycle;
        \fill[line width=2.pt,color=zzttqq,fill=zzttqq,fill opacity=0.10000000149011612] (2.,2.) -- (2.,1.) -- (3.,1.) -- (3.,2.) -- cycle;
        \draw [line width=2.pt,color=zzttqq] (0.,1.)-- (0.,0.);
        \draw [line width=2.pt,color=zzttqq] (0.,0.)-- (1.,0.);
        \draw [line width=2.pt,color=zzttqq] (1.,0.)-- (1.,1.);
        \draw [line width=2.pt,color=zzttqq] (1.,1.)-- (0.,1.);
        \draw [->,line width=2.pt] (0.,0.) -- (2.,1.);
        \draw [line width=2.pt,color=zzttqq] (2.,2.)-- (2.,1.);
        \draw [line width=2.pt,color=zzttqq] (2.,1.)-- (3.,1.);
        \draw [line width=2.pt,color=zzttqq] (3.,1.)-- (3.,2.);
        \draw [line width=2.pt,color=zzttqq] (3.,2.)-- (2.,2.);
        \draw (0.29217687636053036,0.919364566929013) node[anchor=north west] {$A$};
        \draw (2.4222618022745097,1.9010000900158752) node[anchor=north west] {$f(A)$};
        \begin{scriptsize}
        \draw[color=black] (1.398922809252583,0.6242323181577997) node {$z_{0}$};
        \end{scriptsize}
        \end{axis}
        \end{tikzpicture}
    \end{figure}
    Consideriamo un numero complesso $z$ in forma trigonometrica $z=|z|(\cos (\arg z)+ i \sen(\arg z))$. Il numero complesso $iz$ è \[iz=|i||z|\left(\cos\left(\arg z+\frac{\pi}{2}\right)+i\sen\left(\arg z+\frac{\pi}{2}\right)\right)\]
    Quindi è come se fosse stata applicata all'insieme una rotazione di $\frac{\pi}{2}$
    \begin{figure}[h!]
        \centering
        \definecolor{qqwuqq}{rgb}{0.,0.39215686274509803,0.}
        \definecolor{zzttqq}{rgb}{0.6,0.2,0.}
        \definecolor{ududff}{rgb}{0.30196078431372547,0.30196078431372547,1.}
        \begin{tikzpicture}[line cap=round,line join=round,>=triangle 45,x=1.0cm,y=1.0cm]
        \begin{axis}[
        x=1.0cm,y=1.0cm,
        axis lines=middle,
        ymajorgrids=true,
        xmajorgrids=true,
        xmin=-2.754027484353424,
        xmax=2.334785504695761,
        ymin=-0.6771983894106297,
        ymax=2.6654532798667763,
        xtick={-2.0,-1.0,...,2.0},
        ytick={-0.0,1.0,...,2.0},]
        \clip(-2.754027484353424,-0.6771983894106297) rectangle (2.334785504695761,2.6654532798667763);
        \draw [line width=2.pt,color=zzttqq,fill=zzttqq,fill opacity=0.10000000149011612]  (0,0) --  plot[domain=1.1780972450961724:1.5707963267948966,variable=\t]({1.*2.*cos(\t r)+0.*2.*sin(\t r)},{0.*2.*cos(\t r)+1.*2.*sin(\t r)}) -- cycle ;
        \draw [color=qqwuqq](0.4888435380798802,0.8070387323953824) node[anchor=north west] {$\vartheta =\frac{3}{8}\pi$};
        \draw (0.12713869327001168,1.555393583726145) node[anchor=north west] {$B$};
        \draw [line width=2.pt,color=zzttqq,fill=zzttqq,fill opacity=0.10000000149011612]  (0,0) --  plot[domain=1.1780972450961724:1.5707963267948966,variable=\t]({0.*2.*cos(\t r)+-1.*2.*sin(\t r)},{1.*2.*cos(\t r)+0.*2.*sin(\t r)}) -- cycle ;
        \draw (-1.7188032733458691,0.5825322769961536) node[anchor=north west] {$g(B)$};
        \begin{scriptsize}
        \draw [fill=ududff] (0.,0.) circle (2.5pt);
        \draw[color=ududff] (0.08972095070347355,0.22706372261404137) node {$B$};
        \end{scriptsize}
        \end{axis}
        \end{tikzpicture}
    \end{figure}
    Analogamente si ha che siccome 
    \[z^2=|z|^2(\cos (2\arg z)+i\sen(2\arg z)\]
    Il raggio del settore circolare $h(B)$ è $4$ e l'angolo è in $\left]\frac{3}{4}\pi, \pi\right[$
    \begin{proof}
        Dimostriamo che l'insieme \[S=\left\{z:|z|\leq4 \text{ e } \arg z \in \left]\frac{3}{4}\pi, \pi\right[\right\}\] coincide con $h(B)$. Per quanto detto in precedenza $h(b)\subseteq S$, quindi rimane da dimostrare che $h(b)\supseteq S$.\\
        Sia $z\in S: z=\rho (\cos \vartheta + i \sen \vartheta)$, bisogna dimostrare che $\exists w \in B:w^2=z$. Scegliamo \[w=\sqrt{\rho}\left(\cos\frac{\vartheta}{2}+ i \sen \frac{\vartheta}{2}\right)\], per il quale vale $h(w)=z$. Di conseguenza $z\in h(B)$, per cui $h(B)=S$.
    \end{proof}
    Analogamente si dimostra che $h(C)=\{z:1\leq |z|\leq 4\}$.
    \begin{figure}[h!]
        \centering
        \definecolor{qqqqff}{rgb}{0.,0.,1.}
        \begin{tikzpicture}[line cap=round,line join=round,>=triangle 45,x=1.0cm,y=1.0cm]
        \begin{axis}[
        x=1.0cm,y=1.0cm,
        axis lines=middle,
        ymajorgrids=true,
        xmajorgrids=true,
        xmin=-4.5,
        xmax=4.5,
        ymin=-4.5,
        ymax=4.5,
        xtick={-4.0,-3.0,...,4.0},
        ytick={-4.0,-3.0,...,4.0},]
        \clip(-4.5,-4.5) rectangle (4.5,4.5);
        \filldraw [even odd rule, line width=2.pt,color=qqqqff,fill=qqqqff,fill opacity=0.10000000149011612] (0,0) circle (4) (0,0) circle (1);
        \node[color=qqqqff] at (2,2) {$h(C)$};
        \end{axis}
        \end{tikzpicture}
    \end{figure}
%%%%%%%%%%%%%%%%%%%%%%%%%%%%%%%%%%%%%%%%%%%%%%%%%%%%%
%\item 
%\item[\textit{\large Soluzione~}]
%%%%%%%%%%%%%%%%%%%%%%%%%%%%%%%%%%%%%%%%%%%%%%%%%%%%%
\end{enumerate}
\end{document}
