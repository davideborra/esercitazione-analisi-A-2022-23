\documentclass{article}
\usepackage[utf8]{inputenc}
\usepackage{../lambdatex} %disponibile all'indirizzo http://lambdamath.altervista.it/esercizi/lambdatex.sty

\everymath{\displaystyle}
\title{Università degli Studi di Trento - Dipartimento di Matematica\\
CdL in Matematica – a.a. 2022–2023\\ Note esercitazione}
\author{Esercitatore: Simone Verzellesi\thanks{Trascrizione a cura di Davide Borra e Matilde Calabri}}
\date{03 Ottobre 2022}

\begin{document}
\maketitle
\lhead{Note esercitazione}
\chead{Università degli Studi di Trento - Dipartimento di Matematica\\
CdL in Matematica – a.a. 2022–2023}
\rhead{03/10/2022}
\begin{enumerate}[label=\textbf{Esercizio 3.\arabic*.},itemindent=*]
%%%%%%%%%%%%%%%%%%%%%%%%%%%%%%%%%%%%%%%%%%%%%%%%%%%%%
    \item Determinare gli estremi inferiore e superiore ed eventuali massimi e minimi dei seguenti insiemi 
    \begin{itemize}
        \item $A=\left\{\frac{n}{n+1},n\in \N\right\}$
        \item $B=\left\{2(-1)^n-\frac{1}{2n}, n\in \N\setminus \{0\}\right\}$
        \item $C=\left\{\frac{n}{m}: n, m\in \N\setminus\{0\},n>m\right\}$
    \end{itemize}
    \item[\textit{\large Soluzione~}]
    
%%%%%%%%%%%%%%%%%%%%%%%%%%%%%%%%%%%%%%%%%%%%%%%%%%%%%
    \item 
    \item[\textit{\large Soluzione~}]
%%%%%%%%%%%%%%%%%%%%%%%%%%%%%%%%%%%%%%%%%%%%%%%%%%%%%
%\item 
%\item[\textit{\large Soluzione~}]
%%%%%%%%%%%%%%%%%%%%%%%%%%%%%%%%%%%%%%%%%%%%%%%%%%%%%
\end{enumerate}
\end{document}
