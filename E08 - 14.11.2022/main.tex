\documentclass{article}
\usepackage[utf8]{inputenc}
\usepackage{../lambdatex} %disponibile all'indirizzo http://lambdamath.altervista.it/esercizi/lambdatex.sty

\everymath{\displaystyle}
\title{Università degli Studi di Trento - Dipartimento di Matematica\\
CdL in Matematica – a.a. 2022–2023\\ Note esercitazione}
\author{Esercitatore: Simone Verzellesi\thanks{Trascrizione a cura di Davide Borra}}
\date{14 Novembre 2022}

\begin{document}
\maketitle
\lhead{Note esercitazione}
\chead{Università degli Studi di Trento - Dipartimento di Matematica\\
CdL in Matematica – a.a. 2022–2023}
\rhead{14/11/2022}
\setlength{\headheight}{30pt}
\begin{enumerate}[label=\textbf{Esercizio 
    8.\arabic*.},itemindent=*]
%%%%%%%%%%%%%%%%%%%%%%%%%%%%%%%%%%%%%%%%%%%%%%%%%%%%%
    \item Calcolare i seguenti limiti con il teorema di de l'Hôpital:
    \begin{enumerate}
        \item $\lim_{x\to 0^+}\frac{\cos^3x-1}{\log(1+2x)\cosh(x^3)}$
        \item $\lim_{x\to 0^+}\frac{\log(\log(x+1))}{\log x}$
    \end{enumerate}
    \item[\textit{\large Soluzione~}]~
    \begin{enumerate}
        \item Verifichiamo le ipotesi:
        \begin{itemize}
            \item $f(x)=\cos^3x-1\underset{x\to0^+}{\longrightarrow}0$
            \item $g(x)=\log(1+2x)\cosh(x^3)\underset{x\to0^+}{\longrightarrow}0$
            \item $g'(x)=\frac{2}{1+2x}\cosh x^3+3x^2\sinh x^3\log(1+2x)$. Siccome $g(0)=2$, per il teorema della permanenza del segno $\exists b\in \R_{>0}:\forall x \in ]0,b[, g'(x)>0$
        \end{itemize}
        È quindi possibile applicare il teorema di de l'Hôpital
        \[\lim_{x\to 0^+}\frac{\cos^3x-1}{\log(1+2x)\cosh(x^3)}=\left[ \frac{0}{0} \right]\underset{\text{H}}=\lim_{x\to0^+}\frac{-3\sin x \cos^2 x}{\frac{2}{1+2x}\cosh x^3+3x^2\sinh x^3\log(1+2x)}=0\]

        \item Verifichiamo le ipotesi:
        \begin{itemize}
            \item $f(x)=\log(\log(x+1))\underset{x\to0^+}{\longrightarrow}0$
            \item $g(x)=\log x\underset{x\to0^+}{\longrightarrow}0$
            \item $g'(x)\neq 0$ in un intorno destro di $0$, infatti $g'(x)=\frac{1}{x}\neq0~~\forall x\in \R_{>0}$
            \[\lim_{x\to 0^+}\frac{\log(\log(x+1))}{\log x}=\left[ \frac{\infty}{\infty} \right]=\lim_{x\to 0^+}\frac{\frac{1}{x+1}\cdot\frac{1}{\log(x+1)}}{\frac{1}{x}}=\lim_{x\to 0^+}\frac{1}{x+1}\cdot \underbrace{\frac{x}{\log(1+x)}}_{\to 1}\]
        \end{itemize}
    \end{enumerate}
%%%%%%%%%%%%%%%%%%%%%%%%%%%%%%%%%%%%%%%%%%%%%%%%%%%%%
    \item Calcolare il polinomio di Taylor con resto di Peano di
    \begin{enumerate}
        \item $\log x$ di ordine $3$ in $x_0=2$
        \item $\sin(\tg x)+\log (1+\arctg x)$ di ordine $3$ in $x_0=0$
    \end{enumerate}
    \item[\textit{\large Soluzione~}]~
    \begin{enumerate}
        \item Abbiamo due modi: il primo applicando la formula di Taylor mentre il secondo cercando di ricondurci agli sviluppi noti.\\
        \textbf{Modo 1:}
        \[f^{(1)}(x)=\frac{1}{x}~~~~f^{(2)}(x)=-\frac{1}{x^2}~~~~f^{(3)}(x)=+\frac{2}{x^3}\]
        \[\begin{align*}
            \log x & = \log 2+\frac{1}{2}(x-2)-\frac{1}{4\cdot 2!}(x-2)^2+\frac{1}{4\cdot 3!}(x-2)^3+o((x-2)^3)=\\
            &=\log(2)+\frac{1}{2}(x-2)-\frac{1}{8}(x-2)^2+\frac{1}{24}(x-2)^3+o((x-2)^3)=\\
            &=\log 2-\frac{11}{16}+\frac{3}{2}x-\frac{3}{8}x^2+\frac{1}{24}x^3+o((x-2)^3)
        \end{align*}\]
        \textbf{Modo 2:}
        \[f(x)=\log(1+t)=t-\frac{t^2}{2}+\frac{t^3}{3}+o(t^3)\]
        Devo riuscire a ricondurmici:
        \[\log(x)=\log(\frac{x}{2}\cdot 2)=\log2+\log\frac{x}{2}=\log2+\log\left( 1+\left( \frac{x}{2}-1 \right) \right)\]
        Da cui
        \[\log x=\log 2+\left( \frac{x}{2}-1 \right)+\frac{1}{2}\left( \frac{x}{2}-1 \right)^2+\frac{1}{3}\left( \frac{x}{2}-1 \right)^3+o\left(\left( \frac{x}{2}-1 \right)^3\right)\]
        \item Poniamo $t=\tg x$ e $y=\arctg(x)$. In un intorno di $0$, sia $\tg x$ che $\arctg x$ tendono a 0, quindi anche $t$ e $y$. Utilizzando gli sviluppi fondamentali
        \[\sen t+\log(1+ y)=\left[ t-\frac{t^3}{6} +o(t^3)\right]+\left[ y-\frac{y^2}{2} +\frac{y^3}{3}+o(y^3)\right]\]
        Adesso sviluppiamo anche $t$ e $y$:
        \[t=\tg x= x+\frac{x^3}{3}+o(x^3)~~~~~~y=\arctg x=x-\frac{x^3}{3}+o(x^3)\]
        Di conseguenza 
        \[\begin{align*}
            &\sen (\tg x)+\log(1+\arctg x)=\\&=\left[ \left( x+\frac{x^3}{3}+o(x^3) \right)-\frac{\left( x+\frac{x^3}{3}+o(x^3) \right)^3}{6} +o\left(\left( x+\frac{x^3}{3}+o(x^3) \right)^3\right)\right]+\\&+\left[ \left( x-\frac{x^3}{3}+o(x^3) \right)-\frac{\left( x-\frac{x^3}{3}+o(x^3) \right)^2}{2} +\frac{\left( x-\frac{x^3}{3}+o(x^3) \right)^3}{3}+o\left(\left( x-\frac{x^3}{3}+o(x^3) \right)^3\right)\right]=\\&=2x+\frac{x^3}{3}-\frac{1}{2}x^2+o(x^3)
        \end{align*}\]
    \end{enumerate}
%%%%%%%%%%%%%%%%%%%%%%%%%%%%%%%%%%%%%%%%%%%%%%%%%%%%%
\item Calcolare i seguenti limiti utilizzando gli sviluppi di Taylor:
    \begin{enumerate}
        \item $\lim_{x\to 0}\frac{e^{x^2}-\cos x-\frac{3}{2}x^2}{x^4}$
        \item $\lim_{x\to 0}\frac{e^x-1-\log(1-x)}{\tg x-x}$
        \item $\lim_{x\to 0}\frac{x-\sin^2 \sqrt{x}-\sin^2x}{x^2}$
    \end{enumerate}
    \item[\textit{\large Soluzione~}]~
    \begin{enumerate}
        \item $\lim_{x\to 0}\frac{e^{x^2}-\cos x-\frac{3}{2}x^2}{x^4}$
        \\ $e^x=1+x+\frac{x^2}{2}+\frac{x^3}{3!}+o(x^3)~~~\Rarr~~~e^{x^2}=1+x^2+\frac{x^4}{2!}+o(x^4)$
        \\$\cos x=1-\frac{x^2}{2}+\frac{x^4}{24}+o(x^4)$
        \\ Per eliminare la forma indeterminata espando fino al quarto ordine:
        \[
        \begin{align*}
            \lim_{x\to 0}\frac{e^{x^2}-\cos x-\frac{3}{2}x^2}{x^4}&=\lim_{x\to 0}\frac{\left( 1+x^2+\frac{x^4}{2!}+o(x^4) \right)-\left( 1-\frac{x^2}{2}+\frac{x^4}{24}+o(x^4) \right)-\frac{3}{2}x^2}{x^4}=\\&=\lim_{x\to 0}\frac{11}{24}\cdot \frac{\cancel{x^4}}{\cancel{x^4}}+\cancel{\frac{o(x^4)}{x^4}}=\frac{11}{24}
        \end{align*}\]
        \item $\lim_{x\to 0}\frac{e^x-1-\log(1-x)}{\tg x-x}$
        \\ $e^x=1+x+\frac{x^2}{2}+o(x^2)$
        \\ $\log(1+(-x))=-x+\frac{x^2}{2}+o(x^2)$
        \\ $\tg x=x+\frac{x^3}{3}+o(x^3)$
        Sostituisco gli sviluppi
        \[\begin{align*}
            \lim_{x\to 0}\frac{e^x-1-\log(1-x)}{\tg x-x}&=\lim_{x\to 0} \frac{(1+x+o(x)-1-(-x+o(x)))}{(x+\frac{x^3}{3}+o(x^3))-x}=\lim_{x\to 0}\frac{2x+o(x)}{x^3+o(x^3)}=+\infty
        \end{align*}\]
        \item $\lim_{x\to 0}\frac{x-\sin^2 \sqrt{x}-\sin^2x}{x^2}$
        \\Proviamo a sviluppare al primo ordine: $\sin x=x+o(x) ~~\Rarr~~\sen \sqrt{x}=\sqrt{x}+o(\sqrt{x})$
        \[\lim_{x\to 0}\frac{x-(\sqrt{x}+o(\sqrt{x}))^2-(x+o(x))^2}{x^2}=\lim_{x\to 0}\frac{o(x)}{x^2}=\:?\]
        Il primo ordine non è sufficiente: $\sin x=x-\frac{x^3}{3!}+o(x) ~~\Rarr~~\sen \sqrt{x}=\sqrt{x}-\frac{\sqrt{x^3}}{3!}+o(x^\frac{3}{2})$
        \[\begin{align*}
            \lim_{x\to 0}\frac{x-\sin^2 \sqrt{x}-\sin^2x}{x^2}&=\lim_{x\to 0} \frac{x-\left( \sqrt{x}-\frac{\sqrt{x^3}}{3!}+o(x^\frac{3}{2}) \right)^2-\left( x-\frac{x^3}{3!}+o(x) \right)^2}{x^2}=\\&=\lim_{x\to 0}\frac{x-\left( x+\frac{\sqrt{x^6}}{36}-\frac{x^2}{3} +o(x^3)\right)-\left( x^2-\frac{x^4}{3}+\frac{x^6}{36}+o(x^6) \right)}{x^2}=\\&=\lim_{x\to 0}\frac{\frac{x^2}{3}-x^2+\overbrace{\frac{x^3}{36}+o(x^3)}^{o(x^2)}}{x^2}=-\frac{2}{3}
        \end{align*}\]

    \end{enumerate}
%%%%%%%%%%%%%%%%%%%%%%%%%%%%%%%%%%%%%%%%%%%%%%%%%%%%%
\item Dimostrare che $\sin\frac{1}{n}>\frac{1}{n+1}~~~\forall n\in \N_{>0}$
\item[\textit{\large Soluzione~}]~Sviluppiamo con il polinomio di Taylor con resto di Lagrange
\[\sin x=x-\frac{x^3}{6}+\frac{\sin^{(4)}(c)}{4!}x^4~~~\text{(con $c\in\left[ 0,\frac{1}{n} \right]$ )}\]
\[\sin\frac{1}{n}=\frac{1}{n}-\frac{1}{6n^3}+\underbrace{\frac{\sin^{(4)}(c)}{n^44!}}_{>0}\geq\frac{1}{n}-\frac{1}{6n^3}\]
Basta quindi dimostrare che \[\frac{1}{n}-\frac{1}{6n^3}>\frac{1}{n+1}~~\Harr~~6n^2-n-1>0~~\Harr~~n_{1,2}=\frac{1\pm\sqrt{1+24}}{12}=\langle\begin{matrix}\scriptstyle\frac{1}{2}\\\scriptstyle-\frac{1}{3}\end{matrix}\]
\[\begin{cases}
    n<-\frac{1}{3}\lor n>\frac{1}{2}\\
    n\in\N
\end{cases}\Rarr n\geq1\]
QED
%%%%%%%%%%%%%%%%%%%%%%%%%%%%%%%%%%%%%%%%%%%%%%%%%%%%%
\item Calcolare $\sqrt{e}$ con $4$ cifre decimali di precisione.
\item[\textit{\large Soluzione~}]~Sappiamo che \[e^x=\sum_{k=0}^n\frac{x^k}{k!}+\frac{e^c}{(n+1)!}x^{n+1}\]
dove $e^c$ è la derivata $n+1$-esima di $e^x$ calcolata in $x=c$. In particolare
\begin{equation}\sqrt{e}=e^{\frac{1}{2}}=\sum_{k=0}^n\frac{1}{2^kk!}+\frac{e^c}{(n+1)!}\left( \frac{1}{2} \right)^{n+1}\end{equation}
Adesso dobbiamo stabilire per quale valore di $n$ l'errore non influenza le prime $4$ cifre del risultato. Osserviamo che $e<3$ e $c<\frac{1}{2}$ quindi $e^c<\sqrt{3}<2$:
\[\frac{e^c}{(n+1)!2^{n+1}}<\frac{2}{(n+1)}=\frac{1}{2^n(n+1)!}<\frac{1}{10^5}\]
\[10^5<(n+1)!2^n~~~\Rarr~~~n\geq6\]
Sostituendo ora $n$ nella (1) si ricava il valore cercato
%%%%%%%%%%%%%%%%%%%%%%%%%%%%%%%%%%%%%%%%%%%%%%%%%%%%%
\item Determinare per quali $\alpha\in \R$ la serie $\sum_{n= 0}^{+\infty}\left( \frac{2+\alpha}{1-\alpha} \right)^n$ converge e determinarne il valore.
\item[\textit{\large Soluzione~}]~ Osserviamo che si tratta di una serie geometrica di ragione $q:=\frac{2+\alpha}{1-\alpha}$. Quindi $\sum_nq^n$ è
\begin{itemize}
    \item convergente se $|q|<1$
    \item divergente se $q\geq1$
    \item indeterminata se $q\leq 1$
\end{itemize}
Si tratta quindi di risolvere 
\(\left|\frac{2+\alpha}{1-\alpha}\right|<1\)
e si ottiene che la serie è
\begin{itemize}
    \item convergente se $\alpha<-\frac{1}{2}$
    \item divergente se $-\frac{1}{2}\leq \alpha\leq 1$
    \item indeterminata se $\alpha> 1$
\end{itemize}
Inoltre se la serie converge
\[\sum_{n=0}^{+\infty}=\frac{1}{1-q}=\frac{1}{1-\frac{2+\alpha}{1-\alpha}}=\frac{\alpha-1}{2\alpha+1}\]

%%%%%%%%%%%%%%%%%%%%%%%%%%%%%%%%%%%%%%%%%%%%%%%%%%%%%
\item Calcolare \begin{itemize}
    \item $\sum_{n=1}^{+\infty}\frac{n}{(n+1)!}$
    \item $\sum_{n=1}^{+\infty}\frac{1}{n(n+1)(n+2)}$
\end{itemize}
\item[\textit{\large Soluzione~}]~
\begin{itemize}
    \item Cerchiamo di riscrivere la serie per ricondurci a qualcosa di noto
    \[a_n:=\frac{n}{(n+1)!}=\frac{n+1-1}{(n+1=!}=\frac{\cancel{n+1}}{\cancel{(n+1)}n!}-\frac{1}{(n+1)!}=\frac{1}{n!}-\frac{1}{(n+1)!}\]
    \[\sum_{n=1}^{+\infty}\frac{n}{(n+1)!}=\sum_{n=1}^{+\infty}\left(\frac{1}{n!}-\frac{1}{(n+1)!}\right)\]
    Si nota quindi che si tratta di una serie telescopica, per cui si ha che
    \[s_N=\sum_{n=1}^{N}\left(\frac{1}{n!}-\frac{1}{(n+1)!}\right)={1}-\frac{1}{(N+1)!}\]
    Da cui 
    \[\sum_{n=1}^{+\infty}\frac{n}{(n+1)!}=1\lim_{N\to+\infty}\frac{1}{(N+1)!}=1\]
    \item Cerchiamo di riscrivere la serie per ricondurci a qualcosa di noto
    \[a_n:=\frac{1}{n(n+1)(n+2)}=\frac{A}{n(n+1)}+\frac{B}{(n+1)(n+2)}=\frac{(n+2)A+nB}{n(n+1)(n+2)}\]
    \[\begin{cases}
        A+B=0\\
        A=\frac{1}{2}
    \end{cases}~~\Harr~~\begin{cases}
        A=\frac{1}{2}\\
        B=-\frac{1}{2}
    \end{cases}\]
    Di conseguenza riscrivo la serie e mi accorgo che si tratta di una serie telescopica:
    \[\sum_{n=1}^{+\infty}\frac{1}{n(n+1)(n+2)}=\sum_{n=1}^{+\infty}\left( \frac{1}{2n(n+1)}-\frac{1}{2(n+1)(n+1)} \right)=\frac{1}{4}-\lim_{n\to+\infty}\frac{1}{2(n+1)(n+2)}=\frac{1}{4}\]
    
\end{itemize}
%%%%%%%%%%%%%%%%%%%%%%%%%%%%%%%%%%%%%%%%%%%%%%%%%%%%%

\end{enumerate}
\end{document}
